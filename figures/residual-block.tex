
\usetikzlibrary{shapes,arrows,fit,calc,positioning,automata}

\tikzset{XOR/.style={draw,circle,append after command={
        [shorten >=\pgflinewidth, shorten <=\pgflinewidth,]
        (\tikzlastnode.north) edge (\tikzlastnode.south)
        (\tikzlastnode.east) edge (\tikzlastnode.west)
        }
    }
}
\tikzset{line/.style={draw, -latex',shorten <=1bp,shorten >=1bp}}


\begin{figure}
    \centering
    \begin{tikzpicture}

        % for loop over embedding dimension D=6
        \foreach \d in {1,...,6} {
            \node (ii_\d) at (\d*0.7cm, 0.4cm) {};
            \node[draw,fill=black,circle,minimum width=0.2cm, minimum height=0.2cm,inner sep=0] (i_\d) at (\d*0.7cm, 0) {};
            \node[draw,fill=white,XOR,text width=0.3cm,text height=0.3cm,align=center,inner sep=0] (o_\d) at (\d*0.7cm, -10cm) {};
            \node (oo_\d) at (\d*0.7cm, -10.8cm) {};
            \draw[->] (ii_\d) -- (i_\d.north);
            \draw[->,dashed] (i_\d) -- (o_\d.north);
            \draw[->] (o_\d.south) -- (oo_\d);
        }
        \draw[thick,decorate,decoration={calligraphic brace,raise=0.3cm, amplitude=5pt}] (i_1.north) -- node[above=15pt] {Embedding Dimension $D$} (i_6.north);

        \begin{scope}[shift={(2cm, -2cm)}]
            % MLP inputs
            \foreach \d in {1,...,6} {
                \node[draw,fill=white,minimum width=0.7cm, minimum height=0.7cm] (mlp_i_\d) at (\d*0.7cm, 0) {};
            }
            % MLP hidden dimension H=10
            \foreach \h in {1,...,10} {
                \node[draw,fill=white,minimum width=0.7cm, minimum height=0.7cm] (mlp_h1_\h) at (\h*0.7cm - 2*0.7cm, -1.5cm) {};
                \node[draw,fill=white,minimum width=0.7cm, minimum height=0.7cm] (mlp_h2_\h) at (\h*0.7cm - 2*0.7cm, -3cm) {};
                \draw[->] (mlp_h1_\h) -- node[draw,circle,fill=white,inner sep=1pt] {\tiny ϕ} (mlp_h2_\h);
            }
            % MLP outputs
            \foreach \d in {1,...,6} {
                \node[draw,fill=white,minimum width=0.7cm, minimum height=0.7cm] (mlp_o_\d) at (\d*0.7cm, -4.5cm) {};
            }
            % MLP connections
            \foreach \d in {1,...,6} {
                \draw[->] (i_\d) -- (mlp_i_\d.north);
                \foreach \h in {1,...,10} {
                    \draw[->] (mlp_i_\d.south) -- (mlp_h1_\h.north);
                    \draw[->] (mlp_h2_\h.south) -- (mlp_o_\d.north);
                }

                \node[minimum width=0.7cm, minimum height=0.7cm] (attn_\d) at (\d*0.7cm, -6cm) {};
                \draw[->] (mlp_o_\d.south) -- (attn_\d.north);
                \draw[->] (attn_\d.south) -- (o_\d);
            }

            % attention block, just one wide rectangle

            \node[draw,fill=white, minimum width=4.2cm, minimum height=0.7cm,anchor=west] (attention) at (0.35, -6cm) {Attention};

        \end{scope}
    \end{tikzpicture}
    \caption[Typical residual block in transformer]{Typical residual block in transformer (with the $D$ dimension expanded, rather than the $N$ or $M$ dimension as usual.). In the MLP block, the sequence of residual latents are independently projected into a higher-dimensional space, where a non-linearity is applied, and then projected back down to the original dimension. The output of the MLP block is then projected into $Q$, $K$, and $V$ spaces, and the attention operation is applied. Finally, the results are added to the residual stream.}
    \label{fig:residual-block}
\end{figure}
