\chapter{Conclusions}
\label{C:conclusion}

To conclude, the main conclusions from the experiments in \Cref{C:a-o-sampling} and \Cref{C:hand-model} are summarized, along with some reflections on the project as a whole.

\section{Conclusions}

\Cref{C:a-o-sampling} trained a probabilistic model on sequences of pixels in MNIST, and compared the effect of sampling pixels in a variety of orders. The results of this experiment was the discovery that using ``highest-entropy-first'' and ``lowest-entropy-first'' as heuristics for determining the sampling orders resulted in biased samples, of worse visual quality than using a ``random'' sampling order. This is most likely due to imperfections in the model being exacerbated during the sampling process by the entropy hueristic, but not by the random sampling order.

\Cref{C:hand-model} trained both a deterministic model and a probabilistic model on sequences of hand poses from the ManipNet dataset. The deterministic model was able to generate sequences of hand poses that were visually similar to the training data, but the probabilistic model performed very poorly. The conclusion as to why is unclear however, since due to time constraints the experiments needed to investigate why have not been performed. Further investigation is required.

\section{Reflections}

If I had known what I know now at the start of my project, what would I have done differently?

First, training neural networks can be very finnicky. Instead of trying new architectures and model types, I later found that hyperparameters such as the weight initialization, learning rate, and regularization loss terms have a much bigger impact, including on whether the network learns anything reasonable at all. I would have spent more time tuning these hyperparameters, and less time tuning the number of layers, number of neurons, activation functions, and other architecture choices. Relatedly, during the middle of my project, I was working with a custom implementation of a transformer, which was learning poorly. If I was experimenting primarily with these different kinds of hyperparameters, I would have kept using the standard transformer implementation, which would have saved me significant implemenation and debugging time.

Second, training with Masked Sequence Modeling is sufficient to represent the task I was trying to achieve in Chapter 4, but I didn't know this at the outset. I could have used a mostly-pre-implemented data pre-processing pipeline, again saving me significant implementation and debugging time.

Third, running comparison experiments was forever a weak point for me. Doing this project again, I would have spent the additional effort to keep every version of my experiments working in the same codebase simultaneously, so that I could easily switch between them and compare results. Rather than modifying the implementation to run a new experiment (\textit{even if} the current model \& hyperparameters were currently broken), I would have added additional flags and configurations for every experiment, and added some unit tests to make sure that the code still works when these flags are changed.  This would have meant I could have produced more meaningful results in \Cref{C:hand-model}.

\section{Final words}

I have enjoyed working on this thesis, more than I expected at the outset. I have found a passion for learning about deep learning. The project has been challenging but has given me the opportunity to learn an immense amount about machine learning techniques, including modern methods such as transformers. Applying this knowledge to hand motion modeling has also been a challenge of surprising depth. I hope that this thesis has been a useful contribution to the field of hand motion modeling, and that it might be useful to others.
