\chapter{Angles, Joints and Hands}
\label{C:angles-joints-hands}

To a first approximation, the human hand has 23 degrees of freedom.

Each hand has 16 joints - three per digit, plus the wrist. The wrist and the first joint on each digit (the metacarpophalangeal (MCP) joints) can rotate on two axes, and so each have 2 degrees of freedom. The rest (the proximal- and distal-interphalangeal (PIP \& DIP) joints) can only rotate on one axis, and have 1. This naive counting gives 22 degrees of freedom. In addition, we usually consider rotation of the forearm (about the longitudinal axis) to be part of the hand, modeling it as a third degree of freedom of the wrist, which brings the total to 23 degrees of freedom.

Considering that different combinations of joint angles can have very {\fontspec{Symbola}\Large\Large\symbol{"1F91E}} different {\fontspec{Symbola}\Large\Large\symbol{"1F919}} meanings {\fontspec{Symbola}\Large\Large\symbol{"1F44C}},
animators have a lot of work to do when bringing a digital character to life.


However, it is not so straightforward to assign There are a variety of different parameterizations of the joints

An animation of a hand, then consists of a sequence 

% if the citation is the original work then just cite it
% if the citation is not original do e.g. citation

\TODO{ Derivation of $\theta$-MSE minimizing von-mises distribution }

